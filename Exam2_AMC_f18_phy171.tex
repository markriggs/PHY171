\documentclass[letterpaper]{article}
\usepackage[T1]{fontenc}
\usepackage{textcomp}
%\usepackage{lucidabr}
\usepackage{amsmath}
\usepackage{graphicx}
\graphicspath{{./TL_images/}}
%\usepackage[utf8x]{inputenc}
%\usepackage[T1]{fontenc}
\usepackage[box,completemulti]{automultiplechoice}
\begin{document}
\AMCrandomseed{1}
% This argument specifies the number of exams to be printed.
\onecopy{1}{
%%% beginning of the test sheet header:
\noindent{\bf PHY 171 \hfill Exam 2}
\vspace*{.5cm}
\begin{minipage}{.4\linewidth}
\centering\large\bf Fall 2018\end{minipage}
\namefield{\fbox{
    \begin{minipage}{.45\linewidth}
Name:
\vspace*{.5cm}%\dotfill
\vspace*{1mm}
\end{minipage}
}} \hfill

\vspace{1ex}

%%% end of the header


\begin{question}{Average Acceleration}

  Courtney Force topped qualifying with a 3.826-second run at 150.2
  m/s in the NHRA Arizona Nationals. What was her average
  acceleration?
\begin{choices}
\correctchoice{39.26 m/s$^2$}%%
\wrongchoice{39.3 m/s}%%
\wrongchoice{28.33 m/s$^2$}%%
\wrongchoice{36.95 m/s}%%
\wrongchoice{32.8 m/s$^2$}%%
\end{choices}
\end{question}

\begin{question}{Displacement}
A jogger runs down a straight road with an average velocity of 
2.5 m/s for 6.00 minutes. What is her final position if her initial 
position was zero.
\begin{choices}
\correctchoice{900 m}%%
\wrongchoice{15 m}%%
\wrongchoice{14.5 m}%%
\wrongchoice{860 m}%%
\wrongchoice{975 m}%%
\end{choices}
\end{question}

\begin{question}{Newton}
Suppose the net external force exerted on a vaccum cleaner is 23 N
parallel to floor. The mass of the vaccun cleaner is 15 kg. What is
the acceleration?
\begin{choices}
\correctchoice{1.5 m/s$^2$}%%
\wrongchoice{0.652 m/s$^2$}%%
\wrongchoice{0.65 m/s}%%
\wrongchoice{345 m/s$^2$}%%
\wrongchoice{1.53 m/s}\scoring{.75}%%
\end{choices}
\end{question}

\begin{question}{friction}
Suppose a 130 kg wooden crate is resting on a wood floor.
What is the maximum force you can exert horizontally on the crate 
without moving it? For wood on wood $\mu_s=0.5$ and $\mu_k=0.3$.
\begin{choices}
\correctchoice{640 N}%%
\wrongchoice{1300 N}%%
\wrongchoice{380 N}%%
\wrongchoice{65 N}%%
\wrongchoice{720 N}%%
\end{choices}
\end{question}

 
\begin{question}{centripetalacc6}
  Calculate the magnitude of the centripetal acceleration of a car
  following a curve of radius 350m at a speed of 25 m/s.
\begin{choices}
\correctchoice{1.8 m/s$^2$}%%
\wrongchoice{0.14 m/s$^2$}%%
\wrongchoice{0.071 m/s$^2$}%%
\wrongchoice{0.56 m/s$^2$}%%
\wrongchoice{0.98 m/s$^2$}%%
\end{choices}
\end{question}

\begin{question}{omega6}
  A truck with 0.75 m radius tires travels at 48 m/s. What is the
  angular velocity of the rotating tires in radians per second?
\begin{choices}
\correctchoice{64 rad/s}%%
\wrongchoice{36 rad/s}%%
\wrongchoice{32 rad/s}%%
\wrongchoice{58 rad/s}%%
\wrongchoice{67 rad/s}%%
\end{choices}
\end{question}

\begin{question}{energy7}
  What is the power output for a 54 kg woman who runs up a 4.5 m high
  flight is stairs in 3 s, starting from rest but having a final speed
  of 2.5 m/s
\begin{choices}
\correctchoice{850 W}%%
\wrongchoice{2550 W}%%
\wrongchoice{963 W}%%
\wrongchoice{623 W}%%
\wrongchoice{1870 W}%%
\end{choices}
\end{question}

\begin{question}{pe7}
  A hydroelectric power facility converts gravitational potential
  energy of water behind a dam to electrical energy.  What is the
  gravitational potential energy relative to the generators of a lake
  of volume 35 km$^3$, ($3.50 \times 10^{13}$ kg) given that the lake
  has an average height of 25 m above the generators?
\begin{choices}
\correctchoice{$8.58 \times 10^{15}$ J}%%
\wrongchoice{$8.75 \times 10^{14}$ J}%%
\wrongchoice{$8.57 \times 10^{16}$ J}%% 
\wrongchoice{$8.31 \times 10^{15}$ J}%%
\wrongchoice{$6.38 \times 10^{13}$ J}%%
\end{choices}
\end{question}

\begin{question}{momentum8}
  What is the momentum of a fire truck that is $1.25 \times 10^4$ kg
  and is moving at 35 m/s?
  \begin{choices}
\correctchoice{$4.37 \times 10^{5}$ kg m/s}%%
\wrongchoice{$4.38 \times 10^{4}$ kg m/s}%%
\wrongchoice{$1.53 \times 10^{5}$ kg m/s}%%
\wrongchoice{$7.66 \times 10^{6}$ kg m/s}%%
\wrongchoice{$3.98 \times 10^{3}$ kg m/s}%%
\end{choices}
\end{question}

\begin{question}{statics9}
  Two children push in opposite sides of a door during play. Both push
  horizontally and perpendicular to the door. One child pushes with a
  force of 15.0 N at a distance of 0.30 m from the hinges, and the
  second child pushes at a distance of 0.20 m.  What force must the
  second child exert to keep the door from moving? Assume friction if
  negligible.
\begin{choices}
\correctchoice{22.5 N}%%
\wrongchoice{10.0 N}%%
\wrongchoice{16.4 N}%%
\wrongchoice{25.3 N}%%
\wrongchoice{21.2 N}%%
\end{choices}
\end{question}

\begin{question}{collision}
  Train cars are coupled together by being bumped into one
  another. Suppose two loaded train cars are moving toward one
  another, the first having a mass of 112,000 kg and a velocity of 0.3
  m/s, and a second having a mass of 132,000 kg and a velocity of -0.4
  m/s. What is their final velocity?
\begin{choices}
\correctchoice{$-0.0787$ m/s}%%
\wrongchoice{0.786 m/s}%%
\wrongchoice{$-0.354$ m/s}%%
\wrongchoice{0.354 m/s}%%
\wrongchoice{$-7.9$ m/s}%%
\end{choices}
\end{question}

\begin{question}{torque9}
  A person carries a plank of wood 2 m long with one hand pushing down
  on it at one end with a force $F_1$ and the other hand holding it up
  at .35 m from the end of the plank with a force of $F_2$. If the
  plank has a mass of 12 kg and its center of gravity is at the
  middle of the plank, what are the magnitudes of the forces $F_1$ and
  $F_2$?  (Draw a box around your answer)
\AMCOpen{lines=6,dots=false}{
\wrongchoice[w]{w}\scoring{0}%%
\wrongchoice[s]{s}\scoring{0.5}%%
\wrongchoice[h]{h}\scoring{1}%%
\wrongchoice[a]{a}\scoring{1.5}%%
\correctchoice[c]{c}\scoring{2}%%
}
\end{question}

\begin{question}{alpha }
  You have a grindstone that is 95 kg, has a 0.35 m radius and is
  turning at 150 rpm, and you press a steel ax with a radial force of
  25 N.  Assuming the kinetic coefficient of friction between steel
  and stone is 0.2, calculate the angular acceleration of the
  grindstone. Note: $I=\frac{1}{2}mR^2$ \\ (Draw a box around your answer)
\AMCOpen{lines=6,dots=false}{
\wrongchoice[w]{w}\scoring{0}%%
\wrongchoice[s]{s}\scoring{0.5}%%
\wrongchoice[h]{h}\scoring{1}%%
\wrongchoice[a]{a}\scoring{1.5}%%
\correctchoice[c]{c}\scoring{2}%%
}
\end{question}

\begin{question}{KErot }
  Calculate the rotational kinetic energy in a motorcycle wheel if its
  angular velocity is 150 rad/s.  Assume $m=15$ kg, $R_1=0.30$ m and
  $R_2=0.35$ m. Note: $I=\frac{1}{2}m(R_1^2+R_2^2)$ \\ (Draw a box around your answer)
\AMCOpen{lines=5,dots=false}{
\wrongchoice[w]{w}\scoring{0}%%
\wrongchoice[s]{s}\scoring{0.5}%%
\wrongchoice[h]{h}\scoring{1}%%
\wrongchoice[a]{a}\scoring{1.5}%%
\correctchoice[c]{c}\scoring{2}%%
}
\end{question}

\newpage

~

\cleardoublepage

~
}
\end{document}

\begin{question}{ }

\begin{choices}
\correctchoice{}%%
\wrongchoice{}%%
\wrongchoice{}%%
\wrongchoice{}%%
\wrongchoice{}%%
\end{choices}
\end{question}


%%% Local Variables:
%%% mode:
%%% TeX-PDF-mode: t
%%% TeX-master: t
%%% End:
