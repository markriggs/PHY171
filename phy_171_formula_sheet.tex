\documentclass[10pt,landscape]{article}
\usepackage{multicol}
\usepackage{calc}
\usepackage{ifthen}
\usepackage[landscape]{geometry}
\usepackage{hyperref}

% To make this come out properly in landscape mode, do one of the following
% 1.
%  pdflatex latexsheet.tex
%
% 2.
%  latex latexsheet.tex
%  dvips -P pdf  -t landscape latexsheet.dvi
%  ps2pdf latexsheet.ps


% If you're reading this, be prepared for confusion.  Making this was
% a learning experience for me, and it shows.  Much of the placement
% was hacked in; if you make it better, let me know...


% 2008-04
% Changed page margin code to use the geometry package. Also added code for
% conditional page margins, depending on paper size. Thanks to Uwe Ziegenhagen
% for the suggestions.

% 2006-08
% Made changes based on suggestions from Gene Cooperman. <gene at ccs.neu.edu>


% To Do:
% \listoffigures \listoftables
% \setcounter{secnumdepth}{0}


% This sets page margins to .5 inch if using letter paper, and to 1cm
% if using A4 paper. (This probably isn't strictly necessary.)
% If using another size paper, use default 1cm margins.
\ifthenelse{\lengthtest { \paperwidth = 11in}}
	{ \geometry{top=.5in,left=.5in,right=.5in,bottom=.5in} }
	{\ifthenelse{ \lengthtest{ \paperwidth = 297mm}}
		{\geometry{top=1cm,left=1cm,right=1cm,bottom=1cm} }
		{\geometry{top=1cm,left=1cm,right=1cm,bottom=1cm} }
	}

% Turn off header and footer
\pagestyle{empty}
 

% Redefine section commands to use less space
\makeatletter
\renewcommand{\section}{\@startsection{section}{1}{0mm}%
                                {-1ex plus -.5ex minus -.2ex}%
                                {0.5ex plus .2ex}%x
                                {\normalfont\large\bfseries}}
\renewcommand{\subsection}{\@startsection{subsection}{2}{0mm}%
                                {-1explus -.5ex minus -.2ex}%
                                {0.5ex plus .2ex}%
                                {\normalfont\normalsize\bfseries}}
\renewcommand{\subsubsection}{\@startsection{subsubsection}{3}{0mm}%
                                {-1ex plus -.5ex minus -.2ex}%
                                {1ex plus .2ex}%
                                {\normalfont\small\bfseries}}
\makeatother

% Define BibTeX command
\def\BibTeX{{\rm B\kern-.05em{\sc i\kern-.025em b}\kern-.08em
    T\kern-.1667em\lower.7ex\hbox{E}\kern-.125emX}}

% Don't print section numbers
\setcounter{secnumdepth}{0}


\setlength{\parindent}{0pt}
\setlength{\parskip}{0pt plus 0.5ex}


% -----------------------------------------------------------------------

\begin{document}

\raggedright
\footnotesize
\begin{multicols}{3}


% multicol parameters
% These lengths are set only within the two main columns
%\setlength{\columnseprule}{0.25pt}
\setlength{\premulticols}{1pt}
\setlength{\postmulticols}{1pt}
\setlength{\multicolsep}{1pt}
\setlength{\columnsep}{2pt}

\begin{center}
     \Large{\textbf{PHY 171 Formula Sheet}} \\
\end{center}

\subsection{Fundamental SI Units}
\begin{tabular}{@{}ll@{}}
\verb!Length!  & meter (m) \\
\verb!Mass!    & kilogram (kg) \\
\verb!Time! & second (s) \\
\verb!Electric Current!  & ampere (A) \\
\end{tabular}

%\vspace{.25in}

\subsection{Some Derived SI Units}
\begin{tabular}{@{}ll@{}}
\verb!Force!  & \( \mbox{N} = \mbox{kg m}/\mbox{s}^2 \) \\
\verb!Energy! &  \(\mbox{J} = \mbox{kg}\mbox{ m}^2/\mbox{s}^2 \) \\
\verb!Power! &  \( \mbox{W}=\mbox{J}/\mbox{s}\) \\
\verb!Pressure!  & \(\mbox{Pa}=\mbox{N}/\mbox{m}^2 \)  \\
\end{tabular}

%\vspace{.25in}


\subsection{Some Important Constants}
\begin{tabular}{@{}ll@{}}
 &\( G = 6.674 \times 10^{-11} \mbox{ N} \cdot \mbox{m}^2/\mbox{kg}^2 \) \\
  & \( c = 3.00 \times 10^8 \mbox{ m}/\mbox{s}  \)\\
  & \( N_A = 6.02 \times 10^{23} \mbox{ particles}/\mbox{mole} \) \\
  & \(k= 1.38 \times 10^{-23} \mbox{ J}/\mbox{K}\) \\
  & \( g = 9.80 \mbox{ m}/\mbox{s}^2\)\\
  & \( R= 8.31 \mbox{ J}/\mbox{mol} \cdot K\) \\
\end{tabular}

\vspace{.125in}

\( x=\frac{-b \pm \sqrt{b^2-4ac}}{2a} \)

\subsection{Kinematics}
\begin{tabular}{@{}ll@{}}
\verb!Displacement!  & \(\Delta x=x_f-x_0\) \\
\\
\verb!Velocity!    & \( v = \frac{\Delta x}{\Delta t} = \frac{x_f-x_0}{t_f-t_0}\) \\
\\
\verb!Acceleration! & \( a = \frac{\Delta v}{\Delta t} = \frac{v_f-v_0}{t_f-t_0}\) \\
\end{tabular}

%\vspace{.25in}




\begin{tabular}{c}
constant \(a\)  \hfill \qquad\\
\( x=x_0+vt\) \\
\( v = v_0 +at\) \\
\( x=x_0+v_0t+\frac{1}{2}at^2\) \\
\( v^2 = v_0^2+2a(x-x_0) \) \\
  \\
freefall \(a=-g\)  \hfill \qquad\\
\( v=v_0-gt \)\\
\( y = y_0 +v_0 t -\frac{1}{2} g t^2\) \\
\( v^2 =v_0^2 - 2g(y-y_0) \) \\
\end{tabular}

%\vspace{.25in}



\subsection{Projectile motion}
\begin{tabular}{c}
horizontal motion \(a_x=0\) \hfill \qquad \\
\(x=x_0+v_xt\) \\
\( v_x=v_{0x}=v_x \) \\
vertical motion \hfill \qquad\\
\( y = y_0 +v_0 t -\frac{1}{2} g t^2\) \\
\( v_y=v_{0y}-gt \) \\
\( v_y^2 =v_{0y}^2 - 2g(y-y_0) \) \\
\\
\( s=\sqrt{x^2+y^2}\) \\
\(\theta= \tan^{-1}(y/x) \) \\
\( v= \sqrt{v_x^2+v_y^2} \) \\
\(\theta_v= \tan^{-1}(v_y/v_x) \) \\
Maximum height and range \hfill \qquad\\
\( h=\frac{v_{0y}^2}{2g} \) \\
\( R=\frac{v_0^2 \sin 2 \theta_0}{g} \)\\
\end{tabular}

%\vspace{.25in}



\subsection{Dynamics}
\begin{tabular}{c}
\( a =\frac{F}{m} \) \\
\( F=ma \) \\
\( w=mg \) \\
\end{tabular}
\subsection{friction}
\begin{tabular}{c}
\( f_s \le \mu_s N \) \\
\( f_k = \mu_k N \) \\
%Elasticity \hfill \qquad \\
%\( F=k \Delta L \) \\
%\( \Delta L =\frac{F}{k} \) \\
\end{tabular}




\vspace{.25in}

\hrule

\vspace{.25in}

\subsection{Uniform Circular Motion}
\begin{tabular}{c}
\( \Delta \theta= \frac{\Delta s}{r} \) \\
\( 2 \pi \mbox{ rad}= 360^\circ = 1\mbox{ revolution} \) \\
\( \omega = \frac{\Delta \theta}{\Delta t} \) \\
\( v = r \omega \mbox{ or } \omega = \frac{v}{r} \) \\
Centripetal Acceleration \hfill \qquad\\
\( a_c = \frac{v^2}{r} \ ; \ a_c=r \omega^2 \) \\
Centripetal Force \hfill \qquad \\
\( F_c=m a_c\) \\
\( F_c = m \frac{v^2}{r} \) \\
\( F_c = m r \omega^2 \) \\
\( \Delta L =\frac{F}{k} \) \\
\end{tabular}
\subsection{Newton's Univeral Law of Gravitation}
\begin{tabular}{c}
\( F=G\frac{mM}{r^2}\) \\
\end{tabular}

\vspace{.25in}

\subsection{Work and Energy}
\begin{tabular}{c}
\(W = Fd \cos \theta\) \\
\( KE = \frac{1}{2} m v^2 \) \\
\( \Delta PE_g = m g h \) \\
\( W = \Delta KE + \Delta PE \) \\
\( \mbox{PE}_{\mbox{spring}}= \frac{1}{2} k x^2 \) \\
\( P =\frac{W}{t} \) \\
\end{tabular}

\vspace{.25in}

\subsection{Linear Momentum}
\begin{tabular}{c}
\( p = mv\) \\
\( F = \frac{\Delta p}{\Delta t} \) \\
\( p_{\mbox{tot}}= \mbox{ constant} \) \\
\( \mbox{PE}_{\mbox{spring}}= \frac{1}{2} k x^2 \) \\
\end{tabular}


\subsection{Statics and Torque}
\begin{tabular}{c}
\( \tau = r F \sin \theta \) \\
\( \tau = r_\perp F\) \\
\( F_{\mbox{net}}=0 \quad \mbox{net } \tau=0\) \\
\end{tabular}


\subsection{Rotational Motion and Angular Momentum}
\begin{tabular}{c}
\(\omega = \frac{\Delta \theta}{\Delta t} \quad \) 
\(\alpha = \frac{\Delta \omega}{\Delta t} \) \\
\( a_t=r \frac{\Delta \omega}{\Delta t} \quad \) 
\( a_t=r \alpha\) \\
\( x=r\theta \quad \) \( v = r \omega \) \\
\( \theta=\omega_0 t + \frac{1}{2} \alpha t^2\) \\
\( \omega= \omega_0 + \alpha t\) \\
\( \omega^2 = \omega_0^2 + 2 \alpha \theta\) \\
\( \mbox{\footnotesize{net}} \tau = I \alpha \) \\
\( \mbox{KE}_{\mbox{rot}}=\frac{1}{2} I \omega^2\) \\
\end{tabular}

\vspace{.25in}

\hrule

\vspace{.25in}


\subsection{Fluid Statics}
\begin{tabular}{c}
\( \rho = \frac{m}{V} \) \\
\( P = \frac{F}{A} \) \\
\( P = h \rho g \) \\
\( \frac{F_1}{A_1}=\frac{F_2}{A_2} \) \\
\end{tabular}



\subsection{Temperature, Kinetic Theory, \& the Gas Laws }
\begin{tabular}{c}
\( \Delta L = \alpha L \Delta T  \) \\
\(  PV=nRT \quad PV=NkT\) \\
\( \bar{KE}=\frac{1}{2} m \bar{v^2} = \frac{3}{2} k T \)\\
\end{tabular}


\subsection{Heat and Heat Transfer }
\begin{tabular}{c}
\(  Q=mc \Delta T\) \\
\(  \frac{Q}{t}= \frac{kA(T_2-T_1)}{d}\) \\
\( \frac{Q}{t}=\sigma e A T^4\)\\
\(  \) \\
\(  \) \\
\end{tabular}


\subsection{Oscillatory Motions and Waves}
\begin{tabular}{c}
\( F=-kx  \) \\
\( f=\frac{1}{T} \quad T=\frac{1}{f}  \) \\
\( T= 2\pi \sqrt{\frac{L}{g}}  \) \\
\( T= 2\pi \sqrt{\frac{m}{k}}  \) \\
\( f= \frac{1}{2\pi} \sqrt{\frac{K}{m}}  \) \\
\end{tabular}

%\vspace{3 in}

~
\end{multicols}
\end{document}


\subsection{ }
\begin{tabular}{c}
\(  \) \\
\(  \) \\
\(  \) \\
\(  \) \\
\end{tabular}